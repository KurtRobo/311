% Chapter Template

\chapter{Supporting Process Plans} % Main chapter title

\label{Part6Chapter6} % Change X to a consecutive number; for referencing this chapter elsewhere, use \ref{ChapterX}

\lhead{Part \ref{Part6} \nameref{Part6} Chapter \ref{Part6Chapter6}. \emph{\nameref{Part6Chapter6}}} % Change X to a consecutive number; this is for the header on each page - perhaps a shortened title

%----------------------------------------------------------------------------------------
%	SECTION 1
%----------------------------------------------------------------------------------------

\section{Configuration Management Plan}

The system will be managed by the use of automated tools such as Travis-GL, Git and Rin order to manage the configurations of the system.

\begin{description}
	\item{\emph{Git}} the use of the Git repository is to share code amongst the team, so that all members can modify files individually and then commit them so that others may test their changes and get the changes to interact with the rest of the team.
	\item{\emph{Travis-Cl}} A continous building and testing system Travis-Cl will allow for each commit of a feature (during an iteration) to be tested against a set of pre-defined units. This will mean commits will fail to meet a status of passing the test, will not be merged into the master branch
\end{description}
%----------------------------------------------------------------------------------------
%	SECTION 2
%----------------------------------------------------------------------------------------

\section{Evalution Plan}

Due to the short development cycle of the project, a system meeting the core requirements as described in the Software Requirement Specification will meet the success citeria of both the development team and the client.

%----------------------------------------------------------------------------------------
%	SECTION 3
%----------------------------------------------------------------------------------------

\section{Documentation Plan}

The system will be delivered alongside the following documentation

\begin{description}
	\item{\emph{Business Case}} \hfill \\ A document describing and analysing the system and it's position with other systems
	\item{\emph{Project Objectives and Sub-Objectives}} \hfill \\ A list of the project objectives and sub objectives
	\item{\emph{Risk Analysis and Measures}} \hfill \\ A document containing risk management as well as a detailed risk list
	\item{\emph{Effort Estimation}} \hfill \\ A document describing the effort required to develop the system
	\item{\emph{Project Plan}} \hfill \\ A detailed document fully described the plan behind the project.
	\item{\emph{Software Requirement Specifications}} \hfill \\ A document descibing the requirements of the system and factors affecting the requirements
	\item{\emph{Architecture Design}} \hfill \\ A complete architecture design of the system.
	\item{\emph{Test Plans, Cases and Details}} \hfill \\ An accurate description of the testing procedures, the tests cases and the tests implemented in detail.
	\item{\emph{User Manual}} \hfill \\ A document informing the user how to use the software system
	\item{\emph{Meeting Minutes}} \hfill \\ Minutes from all of the project meetings
	\item{\emph{Progress Report}} \hfill \\ A report describing the progress of the development team implementing the project
\end{description}

%----------------------------------------------------------------------------------------
%	SECTION 4
%----------------------------------------------------------------------------------------

\section{Quality Assurance Plan}

I don't know


%----------------------------------------------------------------------------------------
%	SECTION 7
%----------------------------------------------------------------------------------------

\section{Process Improvement Plan}

For the system once all of the Core Requirements are met, the system will strive to meet the High requirements, followed by the medium requirements and if time permits the low requirements.


