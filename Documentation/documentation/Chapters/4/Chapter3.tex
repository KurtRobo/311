% Chapter Template

\chapter{Risk Management} % Main chapter title

\label{Part4Chapter3} % Change X to a consecutive number; for referencing this chapter elsewhere, use \ref{ChapterX}

\lhead{Part \ref{Part4} \nameref{Part4} Chapter \ref{Part4Chapter3}. \emph{\nameref{Part4Chapter3}}} % Change X to a consecutive number;

By providing each member of the development team a duty to report any risk that arises in their development, It can safety be determined the number of risks involved during the development within a 10\% margin of error. Furthermore by discussing these risks with the development team and the client, the risks are able to be classified and from there mitigated or accepted by the development team thus potentially limiting or removing their impact on the project.

The development team will be focusing on prevention or mitigation for the risk where it is possible, however in some cases (such as Flaws in the technology used in the system) these risks will need to be accepted, as the cost of mitigating the specific risk will have a negative impact on the project by its use of resources in mitigation. The use of branches in the git system will allow for the comparison of the system before and after the risk mitigation to assist in determining the impact of the mitigation of the risk on the system.

Finally, the team will assess the effectiveness of the risk prevention during its weekly meeting, ensuring that each risk is given an resonable solution to deal with the issue it presents.  

