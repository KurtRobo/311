% Chapter Template

\chapter{Risks} % Main chapter title

\label{Part4Chapter7} % Change X to a consecutive number; for referencing this chapter elsewhere, use \ref{ChapterX}

\lhead{Part \ref{Part4} \nameref{Part4} Chapter \ref{Part4Chapter7}. \emph{\nameref{Part4Chapter7}}} % Change X to a consecutive number;

%Variables
\newcounter{cSection}
\newcounter{cRisk}[cSection]
%

%----------------------------------------------------------------------------------------
%	SECTION 1
%----------------------------------------------------------------------------------------

\section{Actor Risks}
\FloatBarrier
\label{Development_Risks}
\stepcounter{cSection}

\subsection{Risks Xeta}
\stepcounter{cRisk}

\begin{table}
	\begin{tabularx}{\textwidth}{| X |}
		\hline
		\textbf{Risk Record} \\
	\end{tabularx}
	\begin{tabularx}{\textwidth}{| l | X | l | X |}
		\hline
		\textbf{Risk ID:} & Actor:1 & \textbf{Risk title:} & Development Team Member is added \\
	\end{tabularx}
	\begin{tabularx}{\textwidth}{| l | X | l | X | l | X |}
		\hline
		\textbf{Owner:} & Kurt ROBINSON & \textbf{Date raised:} & 21/MAR/2014 & \textbf{Status} & Accepted \\
	\end{tabularx}
	\begin{tabularx}{\textwidth}{| X |}
		\hline
		\textbf{Risk description:} \\ A new member has been added to the Optimus development team \\
	\end{tabularx}
	\begin{tabularx}{\textwidth}{| X |}
		\hline
		\textbf{Impact description:} \\ The impact causes more division of tasks and requires time taken to fill the new member in with the project \\
	\end{tabularx}
	\begin{tabularx}{\textwidth}{| X |}
		\hline
		\textbf{Recommended risk Acceptance:} \\ As there is not much we can do about a new member being added to the group we must accept it and spend some time filling them in with our development processes and group procedures. This may take some time however once completed we will gain productivity as we have more people to contribute. \\
	\end{tabularx}
	\begin{tabularx}{\textwidth}{| X |}
		\hline
		\textbf{Probability/impact values:} \\
	\end{tabularx}
	\begin{tabularx}{\textwidth}{| l | l | X | X | X |}
		\hline
		 &  \textbf{Probability} & \textbf{Cost} & \textbf{Duration} & \textbf{Quality} \\ \hline
		\textbf{Pre-mitigation} & & & & \\ \hline
		\textbf{Post-mitigation} & & & & \\ \hline \hline
	\end{tabularx}
	\begin{tabularx}{\textwidth}{| l | X | l | X |}
		\hline
		\textbf{Date:} & \textbf{Incident/action} & \textbf{Actor} & \textbf{Outcome/comment} \\ \hline
		 & &  &  \\ \hline
	\end{tabularx}
\end{table}

\begin{table}
	\begin{tabularx}{\textwidth}{| X |}
		\hline
		\textbf{Risk Record} \\
	\end{tabularx}
	\begin{tabularx}{\textwidth}{| l | X | l | X |}
		\hline
		\textbf{Risk ID:} & Actor:2 & \textbf{Risk title:} & Development Team Member leaves \\
	\end{tabularx}
	\begin{tabularx}{\textwidth}{| l | X | l | X | l | X |}
		\hline
		\textbf{Owner:} & Kurt ROBINSON & \textbf{Date raised:} & 21/MAR/2014 & \textbf{Status} & Accepted \\
	\end{tabularx}
	\begin{tabularx}{\textwidth}{| X |}
		\hline
		\textbf{Risk description:} \\ A member of the development team has decided to leave the group \\
	\end{tabularx}
	\begin{tabularx}{\textwidth}{| X |}
		\hline
		\textbf{Impact description:} \\ This will result in their tasks becoming unassigned and the remaining members will need to take on some of these. This will set back development slightly and members will need to work harder. \\
	\end{tabularx}
	\begin{tabularx}{\textwidth}{| X |}
		\hline
		\textbf{Recommended risk Acceptance:} \\ As it is their choice (and likely to have been caused do to reasons not relating to the project) we cannot stop them from leaving. \\
	\end{tabularx}
	\begin{tabularx}{\textwidth}{| X |}
		\hline
		\textbf{Probability/impact values:} \\
	\end{tabularx}
	\begin{tabularx}{\textwidth}{| l | l | X | X | X |}
		\hline
		 &  \textbf{Probability} & \textbf{Cost} & \textbf{Duration} & \textbf{Quality} \\ \hline
		\textbf{Pre-mitigation} & & & & \\ \hline
		\textbf{Post-mitigation} & & & & \\ \hline \hline
	\end{tabularx}
	\begin{tabularx}{\textwidth}{| l | X | l | X |}
		\hline
		\textbf{Date:} & \textbf{Incident/action} & \textbf{Actor} & \textbf{Outcome/comment} \\ \hline
		 & &  &  \\ \hline
	\end{tabularx}
\end{table}

\begin{table}
	\begin{tabularx}{\textwidth}{| X |}
		\hline
		\textbf{Risk Record} \\
	\end{tabularx}
	\begin{tabularx}{\textwidth}{| l | X | l | X |}
		\hline
		\textbf{Risk ID:} & Actor:3 & \textbf{Risk title:} & Development Team Member becomes ill (or must attend to a family matter) \\
	\end{tabularx}
	\begin{tabularx}{\textwidth}{| l | X | l | X | l | X |}
		\hline
		\textbf{Owner:} & Joshua BROWN & \textbf{Date raised:} & 21/MAR/2014 & \textbf{Status} & Accepted \\
	\end{tabularx}
	\begin{tabularx}{\textwidth}{| X |}
		\hline
		\textbf{Risk description:} \\ A member of the development team has become ill \\
	\end{tabularx}
	\begin{tabularx}{\textwidth}{| X |}
		\hline
		\textbf{Impact description:} \\ This will result in less or no work being able to be completed by that individual (depending on the illness). \\
	\end{tabularx}
	\begin{tabularx}{\textwidth}{| X |}
		\hline
		\textbf{Recommended risk Acceptance:} \\ Until that individual is able to recover from that illness their work responsibilities must be split up amongst group members. \\
	\end{tabularx}
	\begin{tabularx}{\textwidth}{| X |}
		\hline
		\textbf{Probability/impact values:} \\
	\end{tabularx}
	\begin{tabularx}{\textwidth}{| l | l | X | X | X |}
		\hline
		 &  \textbf{Probability} & \textbf{Cost} & \textbf{Duration} & \textbf{Quality} \\ \hline
		\textbf{Pre-mitigation} & & & & \\ \hline
		\textbf{Post-mitigation} & & & & \\ \hline \hline
	\end{tabularx}
	\begin{tabularx}{\textwidth}{| l | X | l | X |}
		\hline
		\textbf{Date:} & \textbf{Incident/action} & \textbf{Actor} & \textbf{Outcome/comment} \\ \hline
		  19/APR/2014 & Illness & Kurt ROBINSON & One week \\ \hline
		 28/APR/2014 & Illness & Peter BROWN & One week \\ \hline
	\end{tabularx}
\end{table}

\FloatBarrier

\section{Structure Risks}
\label{Structure_Risks}

\FloatBarrier

\begin{table}
	\begin{tabularx}{\textwidth}{| X |}
		\hline
		\textbf{Risk Record} \\
	\end{tabularx}
	\begin{tabularx}{\textwidth}{| l | X | l | X |}
		\hline
		\textbf{Risk ID:} & Structure:1 & \textbf{Risk title:} & Multiple Documentation Practises  \\
	\end{tabularx}
	\begin{tabularx}{\textwidth}{| l | X | l | X | l | X |}
		\hline
		\textbf{Owner:} & Joshua Brown & \textbf{Date raised:} & 18/MAR/2014 & \textbf{Status} & Mitigated \\
	\end{tabularx}
	\begin{tabularx}{\textwidth}{| X |}
		\hline
		\textbf{Risk description:} \\ Due to multiple software engineers working on this project, It can lead to multiple documentation practises both internal to the system code and to the system documentation.  \\
	\end{tabularx}
	\begin{tabularx}{\textwidth}{| X |}
		\hline
		\textbf{Impact description:} \\ Whilst this may not have an initial effect during the systems development this will have a strong impact on the testing of the system as well as the maintaining of the system even if it is handed by the same developers. \\
	\end{tabularx}
	\begin{tabularx}{\textwidth}{| X |}
		\hline
		\textbf{Recommended risk Mitigation:} \\ In order to best mitigate this risk, The team should have a standards document for their documentation practises. As well as using a \LaTeX document in order to  \\
	\end{tabularx}
	\begin{tabularx}{\textwidth}{| X |}
		\hline
		\textbf{Probability/impact values:} \\
	\end{tabularx}
	\begin{tabularx}{\textwidth}{| l | l | X | X | X |}
		\hline
		 &  \textbf{Probability} & \textbf{Cost} & \textbf{Duration} & \textbf{Quality} \\ \hline
		\textbf{Pre-mitigation} & Significant & Moderate & Ongoing & Low \\ \hline
		\textbf{Post-mitigation} & Low & Low & Ongoing & Low \\ \hline \hline
	\end{tabularx}
	\begin{tabularx}{\textwidth}{| l | X | l | X |}
		\hline
		\textbf{Date:} & \textbf{Incident/action} & \textbf{Actor} & \textbf{Outcome/comment} \\ \hline
		 &  &  &  \\ \hline
	\end{tabularx}%Multiple Documentation Practises
\end{table}

\begin{table}
	\begin{tabularx}{\textwidth}{| X |}
		\hline
		\textbf{Risk Record} \\
	\end{tabularx}
	\begin{tabularx}{\textwidth}{| l | X | l | X |}
		\hline
		\textbf{Risk ID:} & Structure:2 & \textbf{Risk title:} & Multiple Coding Practises  \\
	\end{tabularx}
	\begin{tabularx}{\textwidth}{| l | X | l | X | l | X |}
		\hline
		\textbf{Owner:} & Joshua Brown & \textbf{Date raised:} & 18/MAR/2014 & \textbf{Status} & Mitigated \\
	\end{tabularx}
	\begin{tabularx}{\textwidth}{| X |}
		\hline
		\textbf{Risk description:} \\ Due to multiple software engineers working on this project, It can lead to multiple programming practises implemented in the system.  \\
	\end{tabularx}
	\begin{tabularx}{\textwidth}{| X |}
		\hline
		\textbf{Impact description:} \\ Whilst modular code should in theory be all isolated and shouldn't have to worry about the internals of other modules. It will affect the system if any of the software engineers work on any another engineers module, as well as making maintaining the system more difficult \\
	\end{tabularx}
	\begin{tabularx}{\textwidth}{| X |}
		\hline
		\textbf{Recommended risk Mitigation:} \\ In order to best mitigate this risk, The team should have a standards document for their coding practises  \\
	\end{tabularx}
	\begin{tabularx}{\textwidth}{| X |}
		\hline
		\textbf{Probability/impact values:} \\
	\end{tabularx}
	\begin{tabularx}{\textwidth}{| l | l | X | X | X |}
		\hline
		 &  \textbf{Probability} & \textbf{Cost} & \textbf{Duration} & \textbf{Quality} \\ \hline
		\textbf{Pre-mitigation} & Significant & High & Ongoing & Low \\ \hline
		\textbf{Post-mitigation} & Moderate & Moderate & Ongoing & Low \\ \hline \hline
	\end{tabularx}
	\begin{tabularx}{\textwidth}{| l | X | l | X |}
		\hline
		\textbf{Date:} & \textbf{Incident/action} & \textbf{Actor} & \textbf{Outcome/comment} \\ \hline
		 &  &  &  \\ \hline
	\end{tabularx}%Multiple Coding practises --Struct - Task?
\end{table}

\begin{table}
	\begin{tabularx}{\textwidth}{| X |}
		\hline
		\textbf{Risk Record} \\
	\end{tabularx}
	\begin{tabularx}{\textwidth}{| l | X | l | X |}
		\hline
		\textbf{Risk ID:} & Structure:3 & \textbf{Risk title:} & Governmental Constraints on Development  \\
	\end{tabularx}
	\begin{tabularx}{\textwidth}{| l | X | l | X | l | X |}
		\hline
		\textbf{Owner:} & Kurt ROBINSON & \textbf{Date raised:} & 18/MAR/2014 & \textbf{Status} & Mitigated \\
	\end{tabularx}
	\begin{tabularx}{\textwidth}{| X |}
		\hline
		\textbf{Risk description:} \\ This is a risk that could occur as we are developing a product which may, not to our knowledge, breach government constraints such as using patented material as well as the client giving us requirements which are not entirely legal. \\
	\end{tabularx}
	\begin{tabularx}{\textwidth}{| X |}
		\hline
		\textbf{Impact description:} \\ This would mean that we cannot sell this product as it may be against the law or breaching another companys patented material. We would need to speak with the client to find a workaround that suits them. \\
	\end{tabularx}
	\begin{tabularx}{\textwidth}{| X |}
		\hline
		\textbf{Recommended risk Mitigation:} \\ In order to best mitigate this risk, The team should have a standards document for their documentation practises. As well as using a \LaTeX document in order to  \\
	\end{tabularx}
	\begin{tabularx}{\textwidth}{| X |}
		\hline
		\textbf{Probability/impact values:} \\
	\end{tabularx}
	\begin{tabularx}{\textwidth}{| l | l | X | X | X |}
		\hline
		 &  \textbf{Probability} & \textbf{Cost} & \textbf{Duration} & \textbf{Quality} \\ \hline
		\textbf{Pre-mitigation} & Significant & Moderate & Ongoing & Low \\ \hline
		\textbf{Post-mitigation} & Low & Low & Ongoing & Low \\ \hline \hline
	\end{tabularx}
	\begin{tabularx}{\textwidth}{| l | X | l | X |}
		\hline
		\textbf{Date:} & \textbf{Incident/action} & \textbf{Actor} & \textbf{Outcome/comment} \\ \hline
		 &  &  &  \\ \hline
	\end{tabularx}%Multiple Documentation Practises
\end{table}

\FloatBarrier

\section{Technology Risks}
\label{Technology_Risks}

\FloatBarrier

\begin{table}%New tech (still needs probablity and dates added)
	\begin{tabularx}{\textwidth}{| X |}
		\hline
		\textbf{Risk Record} \\
	\end{tabularx}
	\begin{tabularx}{\textwidth}{| l | X | l | X |}
		\hline
		\textbf{Risk ID:} & Technology:1 & \textbf{Risk title:} & New Technology  \\
	\end{tabularx}
	\begin{tabularx}{\textwidth}{| l | X | l | X | l | X |}
		\hline
		\textbf{Owner:} & Jaime GLENNAN & \textbf{Date raised:} &  & \textbf{Status} & Mitigated \\
	\end{tabularx}
	\begin{tabularx}{\textwidth}{| X |}
		\hline
		\textbf{Risk description:} \\ If a new technology (such as a new version of a technology i.e. patch or a new library which will assist in development) is released during the development of the system. And the team is confident enough to integrate it into the system   \\
	\end{tabularx}
	\begin{tabularx}{\textwidth}{| X |}
		\hline
		\textbf{Impact description:} \\ The new technology may have issues associated with it which are not present in the previous versions of said technology. This results in a potential delay to development in order to fix said flaws.\\
	\end{tabularx}
	\begin{tabularx}{\textwidth}{| X |}
		\hline
		\textbf{Recommended risk Mitigation:} \\ Using consistant versioning for the system, new technologies will only be intergrated into the current version of the system provided. \\
	\end{tabularx}
	\begin{tabularx}{\textwidth}{| X |}
		\hline
		\textbf{Probability/impact values:} \\
	\end{tabularx}
	\begin{tabularx}{\textwidth}{| l | l | X | X | X |}
		\hline
		 &  \textbf{Probability} & \textbf{Cost} & \textbf{Duration} & \textbf{Quality} \\ \hline
		\textbf{Pre-mitigation} & & & & \\ \hline
		\textbf{Post-mitigation} & & & & \\ \hline \hline
	\end{tabularx}
	\begin{tabularx}{\textwidth}{| l | X | l | X |}
		\hline
		\textbf{Date:} & \textbf{Incident/action} & \textbf{Actor} & \textbf{Outcome/comment} \\ \hline
		 &  &  &  \\ \hline
	\end{tabularx}
\end{table}

\begin{table}
	\begin{tabularx}{\textwidth}{| X |}
		\hline
		\textbf{Risk Record}
	\end{tabularx}
	\begin{tabularx}{\textwidth}{| l | X | l | X |}
		\hline
		\textbf{Risk ID:} & Technology:2 & \textbf{Risk title:} & Technology Based Flaws \\
	\end{tabularx}
	\begin{tabularx}{\textwidth}{| l | X | l | X | l | X |}
		\hline
		\textbf{Owner:} & Jaime GLENNAN & \textbf{Date raised:} &  & \textbf{Status} & Mitigated \\
	\end{tabularx}
	\begin{tabularx}{\textwidth}{| X |}
		\hline
		\textbf{Risk description:} \\ A technology used in the system may have a flaw inbuilt in it, this introduces a flaw into the system which will have to either be fixed in the system or the flaw may not be able to be dealt with and will be released with the system \\
	\end{tabularx}
	\begin{tabularx}{\textwidth}{| X |}
		\hline
		\textbf{Impact description:} \\ Whilst modular code should in theory be all isolated and shouldn't have to worry about the internals of other modules. It will affect the system if any of the software engineers work on any another engineers module, as well as making maintaining the system more difficult \\
	\end{tabularx}
	\begin{tabularx}{\textwidth}{| X |}
		\hline
		\textbf{Recommended risk Mitigation:} \\ By isolating the new technologies and testing them on a sandboxed seperate branch of the system it will allow for the technologies to be tested before they are implemented into the main system. Thus allowing to the team to potentially discover any undocumented flaws, before they merge the development branches. \\
	\end{tabularx}
	\begin{tabularx}{\textwidth}{| X |}
		\hline
		\textbf{Probability/impact values:} \\
	\end{tabularx}
	\begin{tabularx}{\textwidth}{| l | l | X | X | X |}
		\hline
		 &  \textbf{Probability} & \textbf{Cost} & \textbf{Duration} & \textbf{Quality} \\ \hline
		\textbf{Pre-mitigation} & & & & \\ \hline
		\textbf{Post-mitigation} & & & & \\ \hline \hline
	\end{tabularx}
	\begin{tabularx}{\textwidth}{| l | X | l | X |}
		\hline
		\textbf{Date:} & \textbf{Incident/action} & \textbf{Actor} & \textbf{Outcome/comment} \\ \hline
		 &  &  &  \\ \hline
	\end{tabularx}%Tech based flaws (still needs probably and dates added)
\end{table}

\begin{table}
	\begin{tabularx}{\textwidth}{| X |}
		\hline
		\textbf{Risk Record} \\
	\end{tabularx}
	\begin{tabularx}{\textwidth}{| l | X | l | X |}
		\hline
		\textbf{Risk ID:} & Technology:3 & \textbf{Risk title:} & Fails due to too many users on the system \\
	\end{tabularx}
	\begin{tabularx}{\textwidth}{| l | X | l | X | l | X |}
		\hline
		\textbf{Owner:} & Peter BROWN & \textbf{Date raised:} &  & \textbf{Status} & Transferred \\
	\end{tabularx}
	\begin{tabularx}{\textwidth}{| X |}
		\hline
		\textbf{Risk description:} \\ If there are more people than first anticipated using Optimus  \\
	\end{tabularx}
	\begin{tabularx}{\textwidth}{| X |}
		\hline
		\textbf{Impact description:} \\ We cannot certainly know how many people will be using this software, as it hosted online it may slow down the whole system if more people are using the software than we first anticipated \\
	\end{tabularx}
	\begin{tabularx}{\textwidth}{| X |}
		\hline
		\textbf{Recommended risk Transfer:} \\ As we are paying for the web hosting we can contact that organisation and upgrade our hosting plan to make our website more capable of taking in larger amounts of traffic thus preventing slowness and/or crashing. That organisation is responsible for ensuring our website is functioning at our intended level \\
	\end{tabularx}
	\begin{tabularx}{\textwidth}{| X |}
		\hline
		\textbf{Probability/impact values:} \\
	\end{tabularx}
	\begin{tabularx}{\textwidth}{| l | l | X | X | X |}
		\hline
		 &  \textbf{Probability} & \textbf{Cost} & \textbf{Duration} & \textbf{Quality} \\ \hline
		\textbf{Pre-mitigation} & & & & \\ \hline
		\textbf{Post-mitigation} & & & & \\ \hline \hline
	\end{tabularx}
	\begin{tabularx}{\textwidth}{| l | X | l | X |}
		\hline
		\textbf{Date:} & \textbf{Incident/action} & \textbf{Actor} & \textbf{Outcome/comment} \\ \hline
		 &  &  &  \\ \hline
	\end{tabularx}
\end{table}

\begin{table}
	\begin{tabularx}{\textwidth}{| X |}
		\hline
		\textbf{Risk Record} \\
	\end{tabularx}
	\begin{tabularx}{\textwidth}{| l | X | l | X |}
		\hline
		\textbf{Risk ID:} & Technology:4 & \textbf{Risk title:} & Clients technology is not compatible with our software \\
	\end{tabularx}
	\begin{tabularx}{\textwidth}{| l | X | l | X | l | X |}
		\hline
		\textbf{Owner:} & James GLENNAN & \textbf{Date raised:} &  & \textbf{Status} & Mitigated \\
	\end{tabularx}
	\begin{tabularx}{\textwidth}{| X |}
		\hline
		\textbf{Risk description:} \\ The software we develop does not function correctly on our clients computer system  \\
	\end{tabularx}
	\begin{tabularx}{\textwidth}{| X |}
		\hline
		\textbf{Impact description:} \\ This will force us to go back and find why they are not compatible and find a solution thus creating more problems \\
	\end{tabularx}
	\begin{tabularx}{\textwidth}{| X |}
		\hline
		\textbf{Recommended risk Mitigation:} \\ We test the software during development testing on the same computer that the client will be using. This will enable us to find where the incompatibility errors arise as soon as the code is added, therefore making it easier to isolate and debug as we go. \\
	\end{tabularx}
	\begin{tabularx}{\textwidth}{| X |}
		\hline
		\textbf{Probability/impact values:} \\
	\end{tabularx}
	\begin{tabularx}{\textwidth}{| l | l | X | X | X |}
		\hline
		 &  \textbf{Probability} & \textbf{Cost} & \textbf{Duration} & \textbf{Quality} \\ \hline
		\textbf{Pre-mitigation} & & & & \\ \hline
		\textbf{Post-mitigation} & & & & \\ \hline \hline
	\end{tabularx}
	\begin{tabularx}{\textwidth}{| l | X | l | X |}
		\hline
		\textbf{Date:} & \textbf{Incident/action} & \textbf{Actor} & \textbf{Outcome/comment} \\ \hline
		 &  &  &  \\ \hline
	\end{tabularx}
\end{table}

\begin{table}
	\begin{tabularx}{\textwidth}{| X |}
		\hline
		\textbf{Risk Record} \\
	\end{tabularx}
	\begin{tabularx}{\textwidth}{| l | X | l | X |}
		\hline
		\textbf{Risk ID:} & Technology:5 & \textbf{Risk title:} & We don't have the tools required to build the system \\
	\end{tabularx}
	\begin{tabularx}{\textwidth}{| l | X | l | X | l | X |}
		\hline
		\textbf{Owner:} & Kurt ROBINSON & \textbf{Date raised:} &  & \textbf{Status} & Transferred \\
	\end{tabularx}
	\begin{tabularx}{\textwidth}{| X |}
		\hline
		\textbf{Risk description:} \\ We do not have the tools required to develop this system  \\
	\end{tabularx}
	\begin{tabularx}{\textwidth}{| X |}
		\hline
		\textbf{Impact description:} \\ If we do not have the tools that are needed to develop the system then we are unable to continue with our development plans. \\
	\end{tabularx}
	\begin{tabularx}{\textwidth}{| X |}
		\hline
		\textbf{Recommended risk Transfer:} \\ This risk is transferred to the University as they are responsible for ensuring that we have all the tools we need to develop such a piece of software provided by the client. If we don't have the tools, they can be given to us upon request. \\
	\end{tabularx}
	\begin{tabularx}{\textwidth}{| X |}
		\hline
		\textbf{Probability/impact values:} \\
	\end{tabularx}
	\begin{tabularx}{\textwidth}{| l | l | X | X | X |}
		\hline
		 &  \textbf{Probability} & \textbf{Cost} & \textbf{Duration} & \textbf{Quality} \\ \hline
		\textbf{Pre-mitigation} & & & & \\ \hline
		\textbf{Post-mitigation} & & & & \\ \hline \hline
	\end{tabularx}
	\begin{tabularx}{\textwidth}{| l | X | l | X |}
		\hline
		\textbf{Date:} & \textbf{Incident/action} & \textbf{Actor} & \textbf{Outcome/comment} \\ \hline
		 &  &  &  \\ \hline
	\end{tabularx}
\end{table}
\FloatBarrier

\section{Task Risks}
\label{Task_Risks}

\FloatBarrier

\begin{table}
	\begin{tabularx}{\textwidth}{| X |}
		\hline
		\textbf{Risk Record} \\
	\end{tabularx}
	\begin{tabularx}{\textwidth}{| l | X | l | X |}
		\hline
		\textbf{Risk ID:} & Tasks:1 & \textbf{Risk title:} & SVN Data changed format \\
	\end{tabularx}
	\begin{tabularx}{\textwidth}{| l | X | l | X | l | X |}
		\hline
		\textbf{Owner:} & Kurt ROBINSON & \textbf{Date raised:} &  & \textbf{Status} & Accepted \\
	\end{tabularx}
	\begin{tabularx}{\textwidth}{| X |}
		\hline
		\textbf{Risk description:} \\ The format of data within the Python SVN has been changed  \\
	\end{tabularx}
	\begin{tabularx}{\textwidth}{| X |}
		\hline
		\textbf{Impact description:} \\ This will mean that our current data import modules may not be compatile with the new format that the data is being stored in within the Python SVN repository. It will cause us to have to redo that module to suit. \\
	\end{tabularx}
	\begin{tabularx}{\textwidth}{| X |}
		\hline
		\textbf{Recommended risk Acceptance:} \\ Our software runs by completing a one time download of the data, however, if this change occurs during the stage where we are developing the import data module then we will need to redesign our system slightly to be compatible with the new data format. \\
	\end{tabularx}
	\begin{tabularx}{\textwidth}{| X |}
		\hline
		\textbf{Probability/impact values:} \\
	\end{tabularx}
	\begin{tabularx}{\textwidth}{| l | l | X | X | X |}
		\hline
		 &  \textbf{Probability} & \textbf{Cost} & \textbf{Duration} & \textbf{Quality} \\ \hline
		\textbf{Pre-mitigation} & Low & & & \\ \hline
		\textbf{Post-mitigation} & & & & \\ \hline \hline
	\end{tabularx}
	\begin{tabularx}{\textwidth}{| l | X | l | X |}
		\hline
		\textbf{Date:} & \textbf{Incident/action} & \textbf{Actor} & \textbf{Outcome/comment} \\ \hline
		 &  &  &  \\ \hline
	\end{tabularx}
\end{table}

\begin{table}
	\begin{tabularx}{\textwidth}{| X |}
		\hline
		\textbf{Risk Record} \\
	\end{tabularx}
	\begin{tabularx}{\textwidth}{| l | X | l | X |}
		\hline
		\textbf{Risk ID:} & Tasks:2 & \textbf{Risk title:} & SVN Data changed format \\
	\end{tabularx}
	\begin{tabularx}{\textwidth}{| l | X | l | X | l | X |}
		\hline
		\textbf{Owner:} & Kurt ROBINSON & \textbf{Date raised:} &  & \textbf{Status} & Mitigated \\
	\end{tabularx}
	\begin{tabularx}{\textwidth}{| X |}
		\hline
		\textbf{Risk description:} \\ The format of data within the Python SVN has been changed  \\
	\end{tabularx}
	\begin{tabularx}{\textwidth}{| X |}
		\hline
		\textbf{Impact description:} \\ This will mean that our current data import modules may not be compatile with the new format that the data is being stored in within the Python SVN repository. It will cause us to have to redo that module to suit. \\
	\end{tabularx}
	\begin{tabularx}{\textwidth}{| X |}
		\hline
		\textbf{Recommended risk Mitgation:} \\ As our software needs to be connected to the SVN repository that may have a change in format, we can mitigate this risk by downloading the entire website to locally stored HTML files. That way we are still able to use our web scrapers to import the data as it was formatted before. \\
	\end{tabularx}
	\begin{tabularx}{\textwidth}{| X |}
		\hline
		\textbf{Probability/impact values:} \\
	\end{tabularx}
	\begin{tabularx}{\textwidth}{| l | l | X | X | X |}
		\hline
		 &  \textbf{Probability} & \textbf{Cost} & \textbf{Duration} & \textbf{Quality} \\ \hline
		\textbf{Pre-mitigation} & Low & & & \\ \hline
		\textbf{Post-mitigation} & & & & \\ \hline \hline
	\end{tabularx}
	\begin{tabularx}{\textwidth}{| l | X | l | X |}
		\hline
		\textbf{Date:} & \textbf{Incident/action} & \textbf{Actor} & \textbf{Outcome/comment} \\ \hline
		 &  &  &  \\ \hline
	\end{tabularx}
\end{table}

\FloatBarrier

\section{Actor-Structure Risks}
\label{Actor-Structure_Risks}

\FloatBarrier

\begin{table}%Lack of dev team communication TBF
	\begin{tabularx}{\textwidth}{| X |}
		\hline
		\textbf{Risk Record} \\
	\end{tabularx}
	\begin{tabularx}{\textwidth}{| l | X | l | X |}
		\hline
		\textbf{Risk ID:} & Actor-Structure:1 & \textbf{Risk title:} & Lack of Development Team Communication \\
	\end{tabularx}
	\begin{tabularx}{\textwidth}{| l | X | l | X | l | X |}
		\hline
		\textbf{Owner:} & Peter BROWN & \textbf{Date raised:} &  & \textbf{Status} & Mitigated \\
	\end{tabularx}
	\begin{tabularx}{\textwidth}{| X |}
		\hline
		\textbf{Risk description:} \\ Insuffient or Ineffective communication between members of the development team. \\
	\end{tabularx}
	\begin{tabularx}{\textwidth}{| X |}
		\hline
		\textbf{Impact description:} \\ It will cause significant delays in the development of the system, due to a lack of communication it can cause task delays as team members may be waiting on completed work as well. \\
	\end{tabularx}
	\begin{tabularx}{\textwidth}{| X |}
		\hline
		\textbf{Recommended risk Mitigation:} \\ If the team uses clear communication channels (i.e. Git for code commit details) as well as having regular face to face meetings it should be able to clear up many of the potential issues that arise.  \\
	\end{tabularx}
	\begin{tabularx}{\textwidth}{| X |}
		\hline
		\textbf{Probability/impact values:} \\
	\end{tabularx}
	\begin{tabularx}{\textwidth}{| l | l | X | X | X |}
		\hline
		 &  \textbf{Probability} & \textbf{Cost} & \textbf{Duration} & \textbf{Quality} \\ \hline
		\textbf{Pre-mitigation} & & & & \\ \hline
		\textbf{Post-mitigation} & & & & \\ \hline \hline
	\end{tabularx}
	\begin{tabularx}{\textwidth}{| l | X | l | X |}
		\hline
		\textbf{Date:} & \textbf{Incident/action} & \textbf{Actor} & \textbf{Outcome/comment} \\ \hline
		 &  &  &  \\ \hline
	\end{tabularx}
\end{table}

\begin{table}
	\begin{tabularx}{\textwidth}{| X |}
		\hline
		\textbf{Risk Record} \\
	\end{tabularx}
	\begin{tabularx}{\textwidth}{| l | X | l | X |}
		\hline
		\textbf{Risk ID:} & Actor-Structure:2 & \textbf{Risk title:} & Lack of Management Communication  \\
	\end{tabularx}
	\begin{tabularx}{\textwidth}{| l | X | l | X | l | X |}
		\hline
		\textbf{Owner:} & Joshua BROWN & \textbf{Date raised:} &  & \textbf{Status} & Mitigated \\
	\end{tabularx}
	\begin{tabularx}{\textwidth}{| X |}
		\hline
		\textbf{Risk description:} \\ Insuffient or Ineffective communication from the Management to the Developers  \\
	\end{tabularx}
	\begin{tabularx}{\textwidth}{| X |}
		\hline
		\textbf{Impact description:} \\ It will cause significant delays in the development of the system, due to a lack of communication it can cause task delays as team members may be waiting on completed work as well. \\
	\end{tabularx}
	\begin{tabularx}{\textwidth}{| X |}
		\hline
		\textbf{Recommended risk Mitigation:} \\ If the management uses clear communication channels with the development team (i.e. IRC for messaging) as well as having regular face to face meetings it should be able to clear up many of the potential issues that arise.  \\
	\end{tabularx}
	\begin{tabularx}{\textwidth}{| X |}
		\hline
		\textbf{Probability/impact values:} \\
	\end{tabularx}
	\begin{tabularx}{\textwidth}{| l | l | X | X | X |}
		\hline
		 &  \textbf{Probability} & \textbf{Cost} & \textbf{Duration} & \textbf{Quality} \\ \hline
		\textbf{Pre-mitigation} & & & & \\ \hline
		\textbf{Post-mitigation} & & & & \\ \hline \hline
	\end{tabularx}
	\begin{tabularx}{\textwidth}{| l | X | l | X |}
		\hline
		\textbf{Date:} & \textbf{Incident/action} & \textbf{Actor} & \textbf{Outcome/comment} \\ \hline
		 & &  &  \\ \hline
	\end{tabularx}% Lack of management team communication
\end{table}

\begin{table}
	\begin{tabularx}{\textwidth}{| X |}
		\hline
		\textbf{Risk Record} \\
	\end{tabularx}
	\begin{tabularx}{\textwidth}{| l | X | l | X |}
		\hline
		\textbf{Risk ID:} & Actor-Structure:3 & \textbf{Risk title:} & Lack of Client Communication  \\
	\end{tabularx}
	\begin{tabularx}{\textwidth}{| l | X | l | X | l | X |}
		\hline
		\textbf{Owner:} & Kurt ROBINSON & \textbf{Date raised:} &  & \textbf{Status} & Mitigated \\
	\end{tabularx}
	\begin{tabularx}{\textwidth}{| X |}
		\hline
		\textbf{Risk description:} \\ Insuffient or Ineffective communication from the Client to the Developers or Vice Versa \\
	\end{tabularx}
	\begin{tabularx}{\textwidth}{| X |}
		\hline
		\textbf{Impact description:} \\ It will cause uncertainty within the group if we do not meet with the client enough regarding the direction we are heading during development of the software \\
	\end{tabularx}
	\begin{tabularx}{\textwidth}{| X |}
		\hline
		\textbf{Recommended risk Mitigation:} \\ If the client and the development team continuously meet (weekly/fortnightly) to get up to date with development and ask questions we will not have an issue regarding project uncertainty. \\
	\end{tabularx}
	\begin{tabularx}{\textwidth}{| X |}
		\hline
		\textbf{Probability/impact values:} \\
	\end{tabularx}
	\begin{tabularx}{\textwidth}{| l | l | X | X | X |}
		\hline
		 &  \textbf{Probability} & \textbf{Cost} & \textbf{Duration} & \textbf{Quality} \\ \hline
		\textbf{Pre-mitigation} & & & & \\ \hline
		\textbf{Post-mitigation} & & & & \\ \hline \hline
	\end{tabularx}
	\begin{tabularx}{\textwidth}{| l | X | l | X |}
		\hline
		\textbf{Date:} & \textbf{Incident/action} & \textbf{Actor} & \textbf{Outcome/comment} \\ \hline
		 & &  &  \\ \hline
	\end{tabularx}
\end{table}

\begin{table}
	\begin{tabularx}{\textwidth}{| X |}
		\hline
		\textbf{Risk Record} \\
	\end{tabularx}
	\begin{tabularx}{\textwidth}{| l | X | l | X |}
		\hline
		\textbf{Risk ID:} & Actor-Structure:4 & \textbf{Risk title:} & New Client \\
	\end{tabularx}
	\begin{tabularx}{\textwidth}{| l | X | l | X | l | X |}
		\hline
		\textbf{Owner:} & James WILSON & \textbf{Date raised:} &  & \textbf{Status} & Accepted \\
	\end{tabularx}
	\begin{tabularx}{\textwidth}{| X |}
		\hline
		\textbf{Risk description:} \\ A new individual has taken over the previous clients position and now makes the decisions regarding the direction of the project and requirements\\
	\end{tabularx}
	\begin{tabularx}{\textwidth}{| X |}
		\hline
		\textbf{Impact description:} \\ This will cause some slowness to development and scheduling as the development team must go through what has been completed so far with the software with the new client. \\
	\end{tabularx}
	\begin{tabularx}{\textwidth}{| X |}
		\hline
		\textbf{Recommended risk Acceptance:} \\ As we are not in any position to choose the position of our clients, we must accept what has happened and move forward to continue development. \\
	\end{tabularx}
	\begin{tabularx}{\textwidth}{| X |}
		\hline
		\textbf{Probability/impact values:} \\
	\end{tabularx}
	\begin{tabularx}{\textwidth}{| l | l | X | X | X |}
		\hline
		 &  \textbf{Probability} & \textbf{Cost} & \textbf{Duration} & \textbf{Quality} \\ \hline
		\textbf{Pre-mitigation} & Low & & & \\ \hline
		\textbf{Post-mitigation} & & & & \\ \hline \hline
	\end{tabularx}
	\begin{tabularx}{\textwidth}{| l | X | l | X |}
		\hline
		\textbf{Date:} & \textbf{Incident/action} & \textbf{Actor} & \textbf{Outcome/comment} \\ \hline
		 & &  &  \\ \hline
	\end{tabularx}
\end{table}

\begin{table}
	\begin{tabularx}{\textwidth}{| X |}
		\hline
		\textbf{Risk Record} \\
	\end{tabularx}
	\begin{tabularx}{\textwidth}{| l | X | l | X |}
		\hline
		\textbf{Risk ID:} & Actor-Structure:5 & \textbf{Risk title:} & Gold Plating \\
	\end{tabularx}
	\begin{tabularx}{\textwidth}{| l | X | l | X | l | X |}
		\hline
		\textbf{Owner:} & Kurt ROBINSON & \textbf{Date raised:} &  & \textbf{Status} & Avoided \\
	\end{tabularx}
	\begin{tabularx}{\textwidth}{| X |}
		\hline
		\textbf{Risk description:} \\ Some team member(s) may focus their time on unnecessary features which were not a requirement and are a waste of programming hours that may be better used towards completing what has been asked of us. For example, rather than making a basic website style at the start a developer may spend a week on designing the website which is not neccesary at that stage of development.  \\
	\end{tabularx}
	\begin{tabularx}{\textwidth}{| X |}
		\hline
		\textbf{Impact description:} \\ This could slow us down in development as we have an individual not following the primary requirements that have been set. \\
	\end{tabularx}
	\begin{tabularx}{\textwidth}{| X |}
		\hline
		\textbf{Recommended risk Avoidance:} \\ To prevent this, team members agree to focus their time only on requirements and the most important elements of the product. If there is time at the end that is when we can add stretch goals and fancy features. \\
	\end{tabularx}
	\begin{tabularx}{\textwidth}{| X |}
		\hline
		\textbf{Probability/impact values:} \\
	\end{tabularx}
	\begin{tabularx}{\textwidth}{| l | l | X | X | X |}
		\hline
		 &  \textbf{Probability} & \textbf{Cost} & \textbf{Duration} & \textbf{Quality} \\ \hline
		\textbf{Pre-mitigation} &  &  &  &  \\ \hline
		\textbf{Post-mitigation} &  &  &  &  \\ \hline \hline
	\end{tabularx}
	\begin{tabularx}{\textwidth}{| l | X | l | X |}
		\hline
		\textbf{Date:} & \textbf{Incident/action} & \textbf{Actor} & \textbf{Outcome/comment} \\ \hline
		 &  &  &  \\ \hline
	\end{tabularx}%Multiple Documentation Practises
\end{table}

\FloatBarrier

\section{Actor-Task Risks}
\label{Actor-Task_Risks}

\FloatBarrier

\begin{table}
	\begin{tabularx}{\textwidth}{| X |}
		\hline
		\textbf{Risk Record} \\
	\end{tabularx}
	\begin{tabularx}{\textwidth}{| l | X | l | X |}
		\hline
		\textbf{Risk ID:} & Actor-Task:1 & \textbf{Risk title:} & Client changes requirements  \\
	\end{tabularx}
	\begin{tabularx}{\textwidth}{| l | X | l | X | l | X |}
		\hline
		\textbf{Owner:} & James WILSON & \textbf{Date raised:} &  & \textbf{Status} & Mitigated \\
	\end{tabularx}
	\begin{tabularx}{\textwidth}{| X |}
		\hline
		\textbf{Risk description:} \\ Client changes project requirements  \\
	\end{tabularx}
	\begin{tabularx}{\textwidth}{| X |}
		\hline
		\textbf{Impact description:} \\ This means potentially going backwards in progress and reworking some parts of the product. The more progress and time made before requirements change then the greater the impact. \\
	\end{tabularx}
	\begin{tabularx}{\textwidth}{| X |}
		\hline
		\textbf{Recommended risk Mitigation:} \\ By keeping in close contact with the client continuously we can find any forecoming requirement changes and the impact of this risk will be greatly reduced. \\
	\end{tabularx}
	\begin{tabularx}{\textwidth}{| X |}
		\hline
		\textbf{Probability/impact values:} \\
	\end{tabularx}
	\begin{tabularx}{\textwidth}{| l | l | X | X | X |}
		\hline
		 &  \textbf{Probability} & \textbf{Cost} & \textbf{Duration} & \textbf{Quality} \\ \hline
		\textbf{Pre-mitigation} &  &  &  &  \\ \hline
		\textbf{Post-mitigation} &  &  &  &  \\ \hline \hline
	\end{tabularx}
	\begin{tabularx}{\textwidth}{| l | X | l | X |}
		\hline
		\textbf{Date:} & \textbf{Incident/action} & \textbf{Actor} & \textbf{Outcome/comment} \\ \hline
		 &  &  &  \\ \hline
	\end{tabularx}%Requirements Change
\end{table}

\FloatBarrier

\section{Actor-Technology Risks}
\label{Actor-Technology_Risks}

\FloatBarrier

\begin{table}
	\begin{tabularx}{\textwidth}{| X |}
		\hline
		\textbf{Risk Record} \\
	\end{tabularx}
	\begin{tabularx}{\textwidth}{| l | X | l | X |}
		\hline
		\textbf{Risk ID:} & Actor-Technology:1 & \textbf{Risk title:} & Development Team have Different Versions of Software/Tools  \\
	\end{tabularx}
	\begin{tabularx}{\textwidth}{| l | X | l | X | l | X |}
		\hline
		\textbf{Owner:} & Joshua BROWN & \textbf{Date raised:} &  & \textbf{Status} & Avoided \\
	\end{tabularx}
	\begin{tabularx}{\textwidth}{| X |}
		\hline
		\textbf{Risk description:} \\ The development team have different versions of software/tools that are being used by each of them to develop the software \\
	\end{tabularx}
	\begin{tabularx}{\textwidth}{| X |}
		\hline
		\textbf{Impact description:} \\ This could potentially cause compatibility issues when working on the same code with different versions of software/tools. For example, one person could write code that is compatible with their system, however it is not compatible on another persons who has a newer or older version. \\
	\end{tabularx}
	\begin{tabularx}{\textwidth}{| X |}
		\hline
		\textbf{Recommended risk Avoidance:} \\ In order to best avoid this risk, each team member has settled on a specific version of software/tool to use and agreed not to update it unless it everybody else is for the decision and if it helps a bug that requires an update to continue development \\
	\end{tabularx}
	\begin{tabularx}{\textwidth}{| X |}
		\hline
		\textbf{Probability/impact values:} \\
	\end{tabularx}
	\begin{tabularx}{\textwidth}{| l | l | X | X | X |}
		\hline
		 &  \textbf{Probability} & \textbf{Cost} & \textbf{Duration} & \textbf{Quality} \\ \hline
		\textbf{Pre-mitigation} &  &  &  &  \\ \hline
		\textbf{Post-mitigation} &  &  &  &  \\ \hline \hline
	\end{tabularx}
	\begin{tabularx}{\textwidth}{| l | X | l | X |}
		\hline
		\textbf{Date:} & \textbf{Incident/action} & \textbf{Actor} & \textbf{Outcome/comment} \\ \hline
		 &  &  &  \\ \hline
	\end{tabularx}%Multiple Documentation Practises
\end{table}

\begin{table}
	\begin{tabularx}{\textwidth}{| X |}
		\hline
		\textbf{Risk Record} \\
	\end{tabularx}
	\begin{tabularx}{\textwidth}{| l | X | l | X |}
		\hline
		\textbf{Risk ID:} & Actor-Technology:2 & \textbf{Risk title:} & Developers lack of knowledge for technology \\
	\end{tabularx}
	\begin{tabularx}{\textwidth}{| l | X | l | X | l | X |}
		\hline
		\textbf{Owner:} & Peter BROWN & \textbf{Date raised:} &  & \textbf{Status} & Mitgated \\
	\end{tabularx}
	\begin{tabularx}{\textwidth}{| X |}
		\hline
		\textbf{Risk description:} \\ The development team is not familiar with the technology being used during development of the software.  \\
	\end{tabularx}
	\begin{tabularx}{\textwidth}{| X |}
		\hline
		\textbf{Impact description:} \\ When the time comes that the team members need to begin development (specifically code), if they do not know the technology required then it could slow them down as they would need to spend time to learn how to use these technologies. \\
	\end{tabularx}
	\begin{tabularx}{\textwidth}{| X |}
		\hline
		\textbf{Recommended risk Mitigation:} \\ We can reduce the impact of needing to use technologies that we lack knowledge in by researching around what we will need to use early on (during the planning phase). This means, when the time comes to use these technologies, the individual will have some knowledge and will only need to be refreshed with their knowledge in order to continue. \\
	\end{tabularx}
	\begin{tabularx}{\textwidth}{| X |}
		\hline
		\textbf{Probability/impact values:} \\
	\end{tabularx}
	\begin{tabularx}{\textwidth}{| l | l | X | X | X |}
		\hline
		 &  \textbf{Probability} & \textbf{Cost} & \textbf{Duration} & \textbf{Quality} \\ \hline
		\textbf{Pre-mitigation} &  &  &  &  \\ \hline
		\textbf{Post-mitigation} &  &  &  &  \\ \hline \hline
	\end{tabularx}
	\begin{tabularx}{\textwidth}{| l | X | l | X |}
		\hline
		\textbf{Date:} & \textbf{Incident/action} & \textbf{Actor} & \textbf{Outcome/comment} \\ \hline
		 &  &  &  \\ \hline
	\end{tabularx}%Multiple Documentation Practises
\end{table}

\FloatBarrier

\section{Structure-Task Risks}
\label{Structure-Task_Risks}

\FloatBarrier

\begin{table}
	\begin{tabularx}{\textwidth}{| X |}
		\hline
		\textbf{Risk Record} \\
	\end{tabularx}
	\begin{tabularx}{\textwidth}{| l | X | l | X |}
		\hline
		\textbf{Risk ID:} & Structure-Task:1 & \textbf{Risk title:} & Unable to complete development before the deadline  \\
	\end{tabularx}
	\begin{tabularx}{\textwidth}{| l | X | l | X | l | X |}
		\hline
		\textbf{Owner:} & James WILSON & \textbf{Date raised:} &  & \textbf{Status} & Reduced \\
	\end{tabularx}
	\begin{tabularx}{\textwidth}{| X |}
		\hline
		\textbf{Risk description:} \\ We have underestimated our time allocation towards certain tasks and were unable to complete development in time \\
	\end{tabularx}
	\begin{tabularx}{\textwidth}{| X |}
		\hline
		\textbf{Impact description:} \\ This would mean that we either have to rush the software and produce a working, unrefined product or we produce an unfinished, functional product. \\
	\end{tabularx}
	\begin{tabularx}{\textwidth}{| X |}
		\hline
		\textbf{Recommended risk Reduction:} \\ We can reduce the chance risk by allocating more team members towards completing difficult tasks so that when it is finished we can go off and get back to our assigned tasks. If we work together on one task that is more difficult than we anticipated then it will be completed earlier and we can work on finising the product quicker. \\
	\end{tabularx}
	\begin{tabularx}{\textwidth}{| X |}
		\hline
		\textbf{Probability/impact values:} \\
	\end{tabularx}
	\begin{tabularx}{\textwidth}{| l | l | X | X | X |}
		\hline
		 &  \textbf{Probability} & \textbf{Cost} & \textbf{Duration} & \textbf{Quality} \\ \hline
		\textbf{Pre-mitigation} & & & & \\ \hline
		\textbf{Post-mitigation} & & & & \\ \hline \hline
	\end{tabularx}
	\begin{tabularx}{\textwidth}{| l | X | l | X |}
		\hline
		\textbf{Date:} & \textbf{Incident/action} & \textbf{Actor} & \textbf{Outcome/comment} \\ \hline
		 & &  &  \\ \hline
	\end{tabularx}% Lack of management team communication
\end{table}

\begin{table}
	\begin{tabularx}{\textwidth}{| X |}
		\hline
		\textbf{Risk Record} \\
	\end{tabularx}
	\begin{tabularx}{\textwidth}{| l | X | l | X |}
		\hline
		\textbf{Risk ID:} & Structure-Task:2 & \textbf{Risk title:} & Change of Requirements \\
	\end{tabularx}
	\begin{tabularx}{\textwidth}{| l | X | l | X | l | X |}
		\hline
		\textbf{Owner:} & James WILSON & \textbf{Date raised:} &  & \textbf{Status} & Mitigated \\
	\end{tabularx}
	\begin{tabularx}{\textwidth}{| X |}
		\hline
		\textbf{Risk description:} \\ The client has changed their requirements for the product \\
	\end{tabularx}
	\begin{tabularx}{\textwidth}{| X |}
		\hline
		\textbf{Impact description:} \\ Depending on how far we are into development this risk could be easy to deal with or a large setback to our work schedule. \\
	\end{tabularx}
	\begin{tabularx}{\textwidth}{| X |}
		\hline
		\textbf{Recommended risk Mitigation:} \\ This can be mitigated by creating a flexible product that would allow for easy addition or removal of features. In the case that this issue ever arose it would take alot less time to recover from compared to if we had build a product that strictly followed the requirements and was not friendly to change. \\
	\end{tabularx}
	\begin{tabularx}{\textwidth}{| X |}
		\hline
		\textbf{Probability/impact values:} \\
	\end{tabularx}
	\begin{tabularx}{\textwidth}{| l | l | X | X | X |}
		\hline
		 &  \textbf{Probability} & \textbf{Cost} & \textbf{Duration} & \textbf{Quality} \\ \hline
		\textbf{Pre-mitigation} & & & & \\ \hline
		\textbf{Post-mitigation} & & & & \\ \hline \hline
	\end{tabularx}
	\begin{tabularx}{\textwidth}{| l | X | l | X |}
		\hline
		\textbf{Date:} & \textbf{Incident/action} & \textbf{Actor} & \textbf{Outcome/comment} \\ \hline
		 & &  &  \\ \hline
	\end{tabularx}% Lack of management team communication
\end{table}

\FloatBarrier

\section{Structure-Technology Risks}
\label{Structure-Technology_Risks}

\FloatBarrier

\FloatBarrier

\section{Task-Technology Risks}
\label{Task-Technology_Risks}

\FloatBarrier

\FloatBarrier
