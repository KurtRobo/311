% Chapter Template

\chapter{Risks} % Main chapter title

\label{Part4Chapter7} % Change X to a consecutive number; for referencing this chapter elsewhere, use \ref{ChapterX}

\lhead{Part \ref{Part4} \nameref{Part4} Chapter \ref{Part4Chapter7}. \emph{\nameref{Part4Chapter7}}} % Change X to a consecutive number;

%Variables
\newcounter{cSection}
\newcounter{cRisk}[cSection]
%

%----------------------------------------------------------------------------------------
%	SECTION 1
%----------------------------------------------------------------------------------------

\section{Actor Risks}
\FloatBarrier
\label{Development_Risks}
\stepcounter{cSection}

\subsection{Risks Xeta}
\stepcounter{cRisk}

\begin{table}
	\begin{tabularx}{\textwidth}{| X |}
		\hline
		\textbf{Risk Record} \\
	\end{tabularx}
	\begin{tabularx}{\textwidth}{| l | X | l | X |}
		\hline
		\textbf{Risk ID:} & counter here & \textbf{Risk title:} & Additions to the Development Team \\
	\end{tabularx}
	\begin{tabularx}{\textwidth}{| l | X | l | X | l | X |}
		\hline
		\textbf{Owner:} & Joshua BROWN & \textbf{Date raised:} &  & \textbf{Status} & \\
	\end{tabularx}
	\begin{tabularx}{\textwidth}{| X |}
		\hline
		\textbf{Risk description:} \\ tyuioprtyuio \\
	\end{tabularx}
	\begin{tabularx}{\textwidth}{| X |}
		\hline
		\textbf{Impact description:} \\ tyuioprtyuio \\
	\end{tabularx}
	\begin{tabularx}{\textwidth}{| X |}
		\hline
		\textbf{Recommended risk Mitigation:} \\ If the team follows  \\
	\end{tabularx}
	\begin{tabularx}{\textwidth}{| X |}
		\hline
		\textbf{Probability/impact values:} \\
	\end{tabularx}
	\begin{tabularx}{\textwidth}{| l | l | X | X | X |}
		\hline
		 &  \textbf{Probability} & \textbf{Cost} & \textbf{Duration} & \textbf{Quality} \\ \hline
		\textbf{Pre-mitigation} & & & & \\ \hline
		\textbf{Post-mitigation} & & & & \\ \hline \hline
	\end{tabularx}
	\begin{tabularx}{\textwidth}{| l | X | l | X |}
		\hline
		\textbf{Date:} & \textbf{Incident/action} & \textbf{Actor} & \textbf{Outcome/comment} \\ \hline
		 & &  &  \\ \hline
	\end{tabularx}
\end{table}



\FloatBarrier

\section{Structure Risks}
\label{Structure_Risks}

\FloatBarrier

\begin{table}
	\begin{tabularx}{\textwidth}{| X |}
		\hline
		\textbf{Risk Record} \\
	\end{tabularx}
	\begin{tabularx}{\textwidth}{| l | X | l | X |}
		\hline
		\textbf{Risk ID:} & Structure:1 & \textbf{Risk title:} & Multiple Documentation Practises  \\
	\end{tabularx}
	\begin{tabularx}{\textwidth}{| l | X | l | X | l | X |}
		\hline
		\textbf{Owner:} & Joshua Brown & \textbf{Date raised:} & 18/MAR/2014 & \textbf{Status} & Completely Mitigated \\
	\end{tabularx}
	\begin{tabularx}{\textwidth}{| X |}
		\hline
		\textbf{Risk description:} \\ Due to multiple software engineers working on this project, It can lead to multiple documentation practises both internal to the system code and to the system documentation.  \\
	\end{tabularx}
	\begin{tabularx}{\textwidth}{| X |}
		\hline
		\textbf{Impact description:} \\ Whilst this may not have an initial effect during the systems development this will have a strong impact on the testing of the system as well as the maintaining of the system even if it is handed by the same developers. \\
	\end{tabularx}
	\begin{tabularx}{\textwidth}{| X |}
		\hline
		\textbf{Recommended risk Mitigation:} \\ In order to best mitigate this risk, The team should have a standards document for their documentation practises. As well as using a \LaTeX document in order to  \\
	\end{tabularx}
	\begin{tabularx}{\textwidth}{| X |}
		\hline
		\textbf{Probability/impact values:} \\
	\end{tabularx}
	\begin{tabularx}{\textwidth}{| l | l | X | X | X |}
		\hline
		 &  \textbf{Probability} & \textbf{Cost} & \textbf{Duration} & \textbf{Quality} \\ \hline
		\textbf{Pre-mitigation} & Significant & Moderate & Ongoing & Low \\ \hline
		\textbf{Post-mitigation} & Low & Low & Ongoing & Low \\ \hline \hline
	\end{tabularx}
	\begin{tabularx}{\textwidth}{| l | X | l | X |}
		\hline
		\textbf{Date:} & \textbf{Incident/action} & \textbf{Actor} & \textbf{Outcome/comment} \\ \hline
		 &  &  &  \\ \hline
	\end{tabularx}%Multiple Documentation Practises 
\end{table}

\begin{table}
	\begin{tabularx}{\textwidth}{| X |}
		\hline
		\textbf{Risk Record} \\
	\end{tabularx}
	\begin{tabularx}{\textwidth}{| l | X | l | X |}
		\hline
		\textbf{Risk ID:} & Structure:2 & \textbf{Risk title:} & Multiple Coding Practises  \\
	\end{tabularx}
	\begin{tabularx}{\textwidth}{| l | X | l | X | l | X |}
		\hline
		\textbf{Owner:} & Joshua Brown & \textbf{Date raised:} & 18/MAR/2014 & \textbf{Status} & Completely Mitigated \\
	\end{tabularx}
	\begin{tabularx}{\textwidth}{| X |}
		\hline
		\textbf{Risk description:} \\ Due to multiple software engineers working on this project, It can lead to multiple programming practises implemented in the system.  \\
	\end{tabularx}
	\begin{tabularx}{\textwidth}{| X |}
		\hline
		\textbf{Impact description:} \\ Whilst modular code should in theory be all isolated and shouldn't have to worry about the internals of other modules. It will affect the system if any of the software engineers work on any another engineers module, as well as making maintaining the system more difficult \\
	\end{tabularx}
	\begin{tabularx}{\textwidth}{| X |}
		\hline
		\textbf{Recommended risk Mitigation:} \\ In order to best mitigate this risk, The team should have a standards document for their coding practises  \\
	\end{tabularx}
	\begin{tabularx}{\textwidth}{| X |}
		\hline
		\textbf{Probability/impact values:} \\
	\end{tabularx}
	\begin{tabularx}{\textwidth}{| l | l | X | X | X |}
		\hline
		 &  \textbf{Probability} & \textbf{Cost} & \textbf{Duration} & \textbf{Quality} \\ \hline
		\textbf{Pre-mitigation} & Significant & High & Ongoing & Low \\ \hline
		\textbf{Post-mitigation} & Moderate & Moderate & Ongoing & Low \\ \hline \hline
	\end{tabularx}
	\begin{tabularx}{\textwidth}{| l | X | l | X |}
		\hline
		\textbf{Date:} & \textbf{Incident/action} & \textbf{Actor} & \textbf{Outcome/comment} \\ \hline
		 &  &  &  \\ \hline
	\end{tabularx}%Multiple Coding practises --Struct - Task?
\end{table}

\FloatBarrier

\section{Technology Risks}
\label{Technology_Risks}

\FloatBarrier

\begin{table}%New tech (still needs probablity and dates added)
	\begin{tabularx}{\textwidth}{| X |}
		\hline
		\textbf{Risk Record} \\
	\end{tabularx}
	\begin{tabularx}{\textwidth}{| l | X | l | X |}
		\hline
		\textbf{Risk ID:} & Technology:1 & \textbf{Risk title:} & New Technology  \\
	\end{tabularx}
	\begin{tabularx}{\textwidth}{| l | X | l | X | l | X |}
		\hline
		\textbf{Owner:} & Jaime GLENNAN & \textbf{Date raised:} &  & \textbf{Status} & \\
	\end{tabularx}
	\begin{tabularx}{\textwidth}{| X |}
		\hline
		\textbf{Risk description:} \\ If a new technology (such as a new version of a technology i.e. patch or a new library which will assist in development) is released during the development of the system. And the team is confident enough to integrate it into the system   \\
	\end{tabularx}
	\begin{tabularx}{\textwidth}{| X |}
		\hline
		\textbf{Impact description:} \\ The new technology may have issues associated with it which are not present in the previous versions of said technology. This results in a potential delay to development in order to fix said flaws.\\
	\end{tabularx}
	\begin{tabularx}{\textwidth}{| X |}
		\hline
		\textbf{Recommended risk Mitigation:} \\ Using consistant versioning for the system, new technologies will only be intergrated into the current version of the system provided. \\
	\end{tabularx}
	\begin{tabularx}{\textwidth}{| X |}
		\hline
		\textbf{Probability/impact values:} \\
	\end{tabularx}
	\begin{tabularx}{\textwidth}{| l | l | X | X | X |}
		\hline
		 &  \textbf{Probability} & \textbf{Cost} & \textbf{Duration} & \textbf{Quality} \\ \hline
		\textbf{Pre-mitigation} & & & & \\ \hline
		\textbf{Post-mitigation} & & & & \\ \hline \hline
	\end{tabularx}
	\begin{tabularx}{\textwidth}{| l | X | l | X |}
		\hline
		\textbf{Date:} & \textbf{Incident/action} & \textbf{Actor} & \textbf{Outcome/comment} \\ \hline
		 &  &  &  \\ \hline
	\end{tabularx}
\end{table}

\begin{table}
	\begin{tabularx}{\textwidth}{| X |}
		\hline
		\textbf{Risk Record}
	\end{tabularx}
	\begin{tabularx}{\textwidth}{| l | X | l | X |}
		\hline
		\textbf{Risk ID:} & Technology:2 & \textbf{Risk title:} & Technology Based Flaws \\
	\end{tabularx}
	\begin{tabularx}{\textwidth}{| l | X | l | X | l | X |}
		\hline
		\textbf{Owner:} & Jaime GLENNAN & \textbf{Date raised:} &  & \textbf{Status} &  \\
	\end{tabularx}
	\begin{tabularx}{\textwidth}{| X |}
		\hline
		\textbf{Risk description:} \\ A technology used in the system may have a flaw inbuilt in it, this introduces a flaw into the system which will have to either be fixed in the system or the flaw may not be able to be dealt with and will be released with the system \\
	\end{tabularx}
	\begin{tabularx}{\textwidth}{| X |}
		\hline
		\textbf{Impact description:} \\ Whilst modular code should in theory be all isolated and shouldn't have to worry about the internals of other modules. It will affect the system if any of the software engineers work on any another engineers module, as well as making maintaining the system more difficult \\
	\end{tabularx}
	\begin{tabularx}{\textwidth}{| X |}
		\hline
		\textbf{Recommended risk Mitigation:} \\ By isolating the new technologies and testing them on a sandboxed seperate branch of the system it will allow for the technologies to be tested before they are implemented into the main system. Thus allowing to the team to potentially discover any undocumented flaws, before they merge the development branches. \\
	\end{tabularx}
	\begin{tabularx}{\textwidth}{| X |}
		\hline
		\textbf{Probability/impact values:} \\
	\end{tabularx}
	\begin{tabularx}{\textwidth}{| l | l | X | X | X |}
		\hline
		 &  \textbf{Probability} & \textbf{Cost} & \textbf{Duration} & \textbf{Quality} \\ \hline
		\textbf{Pre-mitigation} & & & & \\ \hline
		\textbf{Post-mitigation} & & & & \\ \hline \hline
	\end{tabularx}
	\begin{tabularx}{\textwidth}{| l | X | l | X |}
		\hline
		\textbf{Date:} & \textbf{Incident/action} & \textbf{Actor} & \textbf{Outcome/comment} \\ \hline
		 &  &  &  \\ \hline
	\end{tabularx}%Tech based flaws (still needs probably and dates added)
\end{table}

\FloatBarrier

\section{Task Risks}
\label{Task_Risks}

\FloatBarrier

\FloatBarrier

\section{Actor-Structure Risks}
\label{Actor-Structure_Risks}

\FloatBarrier

\begin{table}%Lack of dev team communication TBF
	\begin{tabularx}{\textwidth}{| X |}
		\hline
		\textbf{Risk Record} \\
	\end{tabularx}
	\begin{tabularx}{\textwidth}{| l | X | l | X |}
		\hline
		\textbf{Risk ID:} & Actor-Structure:1 & \textbf{Risk title:} & Lack of Development Team Communication \\
	\end{tabularx}
	\begin{tabularx}{\textwidth}{| l | X | l | X | l | X |}
		\hline
		\textbf{Owner:} & Peter BROWN & \textbf{Date raised:} &  & \textbf{Status} &  \\
	\end{tabularx}
	\begin{tabularx}{\textwidth}{| X |}
		\hline
		\textbf{Risk description:} \\ Insuffient or Ineffective communication between members of the development team. \\
	\end{tabularx}
	\begin{tabularx}{\textwidth}{| X |}
		\hline
		\textbf{Impact description:} \\ It will cause significant delays in the development of the system, due to a lack of communication it can cause task delays as team members may be waiting on completed work as well. \\
	\end{tabularx}
	\begin{tabularx}{\textwidth}{| X |}
		\hline
		\textbf{Recommended risk Mitigation:} \\ If the team uses clear communication channels (i.e. Git for code commit details) as well as having regular face to face meetings it should be able to clear up many of the potential issues that arise.  \\
	\end{tabularx}
	\begin{tabularx}{\textwidth}{| X |}
		\hline
		\textbf{Probability/impact values:} \\
	\end{tabularx}
	\begin{tabularx}{\textwidth}{| l | l | X | X | X |}
		\hline
		 &  \textbf{Probability} & \textbf{Cost} & \textbf{Duration} & \textbf{Quality} \\ \hline
		\textbf{Pre-mitigation} & & & & \\ \hline
		\textbf{Post-mitigation} & & & & \\ \hline \hline
	\end{tabularx}
	\begin{tabularx}{\textwidth}{| l | X | l | X |}
		\hline
		\textbf{Date:} & \textbf{Incident/action} & \textbf{Actor} & \textbf{Outcome/comment} \\ \hline
		 &  &  &  \\ \hline
	\end{tabularx}
\end{table}

\begin{table}
	\begin{tabularx}{\textwidth}{| X |}
		\hline
		\textbf{Risk Record} \\
	\end{tabularx}
	\begin{tabularx}{\textwidth}{| l | X | l | X |}
		\hline
		\textbf{Risk ID:} & Actor-Structure:2 & \textbf{Risk title:} & Lack of Management communication  \\
	\end{tabularx}
	\begin{tabularx}{\textwidth}{| l | X | l | X | l | X |}
		\hline
		\textbf{Owner:} & Joshua BROWN & \textbf{Date raised:} &  & \textbf{Status} & \\
	\end{tabularx}
	\begin{tabularx}{\textwidth}{| X |}
		\hline
		\textbf{Risk description:} \\ Insuffient or Ineffective communication from the Management to the Developers  \\
	\end{tabularx}
	\begin{tabularx}{\textwidth}{| X |}
		\hline
		\textbf{Impact description:} \\ It will cause significant delays in the development of the system, due to a lack of communication it can cause task delays as team members may be waiting on completed work as well. \\
	\end{tabularx}
	\begin{tabularx}{\textwidth}{| X |}
		\hline
		\textbf{Recommended risk Mitigation:} \\ If the management uses clear communication channels with the development team (i.e. IRC for messaging) as well as having regular face to face meetings it should be able to clear up many of the potential issues that arise.  \\
	\end{tabularx}
	\begin{tabularx}{\textwidth}{| X |}
		\hline
		\textbf{Probability/impact values:} \\
	\end{tabularx}
	\begin{tabularx}{\textwidth}{| l | l | X | X | X |}
		\hline
		 &  \textbf{Probability} & \textbf{Cost} & \textbf{Duration} & \textbf{Quality} \\ \hline
		\textbf{Pre-mitigation} & & & & \\ \hline
		\textbf{Post-mitigation} & & & & \\ \hline \hline
	\end{tabularx}
	\begin{tabularx}{\textwidth}{| l | X | l | X |}
		\hline
		\textbf{Date:} & \textbf{Incident/action} & \textbf{Actor} & \textbf{Outcome/comment} \\ \hline
		 & &  &  \\ \hline
	\end{tabularx}% Lack of management team communication
\end{table}



\FloatBarrier








