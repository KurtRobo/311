% Chapter Template

\chapter{Introduction} % Main chapter title

\label{Part1Chapter1} % Change X to a consecutive number; for referencing this chapter elsewhere, use \ref{ChapterX}

\lhead{Part \ref{Part2} \nameref{Part2} Chapter \ref{Part1Chapter1}. \emph{\nameref{Part1Chapter1}}} % Change X to a consecutive number;


%----------------------------------------------------------------------------------------
%	SECTION 1
%----------------------------------------------------------------------------------------

\section{Purpose}

The purpose of this document is to collate the relevant documentation created in the process of developing the system.

The intended readers of this Report are The development team, the users and the clients. 

%----------------------------------------------------------------------------------------
%	SECTION 2
%----------------------------------------------------------------------------------------

\section{Scope}

This report is the collation of the documents which were used to assist in the development of the software solution.
%

%----------------------------------------------------------------------------------------
%	SECTION 4
%----------------------------------------------------------------------------------------

\section{Overview}

The sections and subsections following this Introduction contain:

\begin{itemize}
	\item Part \ref{Part2}: \nameref{Part2} which is describes the systems position and comparable systems
	\item Part \ref{Part3}: \nameref{Part3} which is describes the projects objectives and factors relating to them
	\item Part \ref{Part4}: \nameref{Part4} which analyses the risks, how to manage the risks.
	\item Part \ref{Part5}: \nameref{Part5} which analyses the effort required to implement the system
	\item Part \ref{Part6}: \nameref{Part6} A detailed document fully described the plan behind the project.
	\item Part \ref{Part7}: \nameref{Part7}  A document descibing the requirements of the system and factors affecting the requirements
	\item Part \ref{Part8}: \nameref{Part8}  A complete architecture design of the system.
	\item Part \ref{Part9}: \nameref{Part9}  An accurate description of the testing procedures, the tests cases and the tests implemented in detail.
	\item Part \ref{Part10}: \nameref{Part10}  A document informing the user how to use the software system